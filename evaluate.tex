
\subsection{Approaches}
There are two different ways to evaluate an ontology, qualitative and quantitative approach. With a qualitative approach, it is difficult to know who the right persons to evaluate are. There could be the user that should use the system, there could be the domain expert or someone else who has competence at the area. It is also hard to develop automated tests to the ontologies. The quantitative evaluating of an ontology consider the effectivity of an ontology in a context of an application and is the most common one. There is one more approach that concerns the congruence between an ontology and a domain of knowledge. When a new developed ontology is used together with an existing "equal" ontology at a domain of knowledge, and the results differs, it is not obvious to find the causes. It can be that the corpus is inappropriate or there is a real difference in the knowledge showed in the corpus and the existing ontology.\cite{brewster}

\subsection{Methodologies}
The methodologies for evaluating an ontology is not the same as methodologies for ontology engineering. In evaluation, they provide a framework for defining well-suited methods for evaluating ontologies. Here I will present two ontology-evaluating methodologies: \cite{yu}

\subsubsection{OntoClean}
OntoClean is a methodology for analyzing ontologies that is based on formal, domain-independent properties of classes. By using OntoClean, can help an ontology to meet the evaluation criterion of correctness. Here correctness means if the entities and properties in an ontology correct give a picture of the world that is modeled. The way OntoClean does this is by introducing meta-properties  to capture different characteristics of classes and constraints by those meta-properties. That help to consider the correct usage of the subsumption relation between classes in an ontology.

The meta-properties in OntoClean is identity, unity, rigidity and dependence. Identity is normal in many fields, as metaphysics and database conceptual modeling. In this cases there is an accepted practice to make a primary key for the rows in a table. Each row has a unique primary key, so if "more" than one row have the same primary key, they are the same row. In OntoClean, identity criteria are some classes of entities, called sortals. This is a class all of whoes instances that are identified in the same way. In information systems, criteria like this are often outer, like universally unique id, but in the ontology world, this is not interesting. Identity criteria should be informative and should be helping the users to understand what a class means. The next meta-properties is unity, that are some properties that only contains of individuals that are wholes. In OntoClean, wholes are individuals all of whose parts are related to each other, and only to each other, by some distinguished relation. The third meta-properties can be described by an example; If I have long hair one day and then take a cut off the hair the next day, yet I am the same entity at both times. How is it possible for me to be the same if I have changed? This is one of many logical approaches to this dilemma, the most usual is to look at some properties to be essential and an essential property of an entity cannot change, and there are for these kind of properties Leibniz's law hold. The properties of an entity that are non-essential can change, but they cannot be involved in the identity. The last meta-property in the OntoClean methodology is dependence. This mean that if a property should be dependent, each of the instances of it implies the existence of another entity.
\cite{yu,website:wikipediaontoclean}

\subsubsection{OntoMetric}
The OntoMetric methodology is using application constraints as the basis for ontology selection. In OntoMetric there are five dimentions at the top level of the taxonomy, that are: content, language, methodology that is used to develop the ontology, tools that are used to develop the ontology and the cost to utilise the ontology. There are associated a set of factors to each of the dimension and for each factor. To get the characteristics it is needed to take this from existing work and include design qualities, ontology evaluation criteria, cost and language characteristics. The OntoMetric methodology use this steps:
\begin{enumerate}%Utdyp mer!
	\item Analyse project aim
	\item Obtain a customised a multilevel tree of characteristics (MTC). This should be based on a set of objectives from the project aim.
	\item Weight up each characteristics against each other. Two and two characteristics are weighted against each other to show the importance of one characteristics over another. A given weight wt is given for each characteristics to assign them against each other. A comparison matrix is made of the pairwise comparison and the eigenvectors are calculated from this matrix.
	\item Assign linguistic score for each characteristics of a candidate ontology. 
	\item Select the most suitable ontology. How close the characteristics of each candidate ontology is evaluated. This is showed by comparing vectors of the weight wt and wc of the candidate ontology c and the modified taxonomy of characteristics in the customized MTC.
\end{enumerate}
There are some limitations in the OntoMetric methodology too, like that the determining the customised MTC for the ontology selection is depending at the manual specification, and this can be subjective or inconsistent. The list of characteristics that is used for evaluating content is limited. The linguistic scale is up to the user to assign values of an ontology characteristics, so there is not used specific measurements. The OntoMetric methodology can only be used to check which ontology that is best fit from a set of candidates ontologies. \cite{yu}
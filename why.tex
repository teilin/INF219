In the World-Wide Web, there has started to be common to develop ontologies to categorization of products. There are also more and more fields that have developed standarized ontologies to make it easier to share a common understanding of the information. In the World-Wide Web, the WWW Consortium (W3C)\footnote{International community where member organizations, full-time staff and the public collaborate to develop Web-standards \cite{website:w3c}} is developing the "Resource Description Framework", that is a language for encoding knowledge on Web pages. This makes it understandable for electronic agents that is searching for information. An ontology is useful because it defines a common vocabulary in the field so different researchers easier can share information in a domain and the understanding is better. The United Nations Development Program and Dun \& Bradstreet have developed the UNSPSC ontology which contain terminology for products and services (http://www.unspsc.org/).

By developing an ontology it also enable the reuse of the domain of knowledge. If some researchers develop an ontology, other researchers that are developing or need an ontology that is similar, can just reuse it.

Developing an ontology of the domain is often not the aim, but developing an ontology to define a set of data and the structure for other programs to use. That can be software agents or web applications.\cite{website:standford}